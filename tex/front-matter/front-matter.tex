\documentclass[../thesis/thesis.tex]{subfiles}
\begin{document}

\cleardoublepage
\setcounter{page}{1}

\chapter*{Declaration of Authorship}
This thesis and the work to which it refers are the results of my own efforts. Any ideas, data, images or text resulting from the work of others (whether published or unpublished) are fully identified as such within the work and attributed to their originator in the text, bibliography or in footnotes. This thesis has not been submitted in whole or in part for any other academic degree or professional qualification. I
agree that the University has the right to submit my work to the plagiarism detection service TurnitinUK for originality checks. Whether or not drafts have been so assessed, the University reserves the right to require an electronic version of the final document (as submitted) for assessment as above.
\\[48pt]
\noindent\begin{tabular}{ll}
\makebox[2in]{\hrulefill} & \makebox[2in]{\hrulefill} \\ Laurence Stant (Author) & Date \\
\end{tabular}

\chapter*{}
\emph{``What error drives our eyes and ears amiss? Until I know this sure uncertainty I'll entertain the offered fallacy."}
\begin{flushright}
William Shakespeare, The Comedy of Errors
\end{flushright}
\vspace{1cm}
\emph{``That's right!" shouted Vroomfondel, ``we demand rigidly defined areas of doubt and uncertainty!"}
\begin{flushright}
Douglas Adams, The Hitchikers Guide to the Galaxy
\end{flushright}

\chapter*{Abstract}
\renewcommand{\baselinestretch}{1.5}\selectfont
To support the responsible implementation of next-generation wireless communications networks such as 5G, the efficiency of power amplifiers located in both base-stations and mobile handsets must be improved. Significant efficiency gains can be obtained by using nonlinear amplifier techniques, however these cause undesired distortion to the signal. Methods used to mitigate these effects rely on accurate models extracted from the internal transistors, which circuit simulators interrogate to predict the performance of new amplifier designs. This thesis presents the first evaluation of measurement uncertainty propagated into a nonlinear behavioural model, X-parameters, and used within a circuit simulator to provide confidence in the results. This uncertainty evaluation can also reveal the relative uncertainty contributions from different aspects of the measurement setup, the knowledge of which can be used to make informed improvements in manufacturing test laboratories. The evaluation was tested on a millimetre-wave amplifier designed for communications use, which showed encouraging results when simulated in a test circuit to provide figures for gain and PAE. During development of this uncertainty evaluation, a standard guidance document was reviewed and found to contain ambiguities which significantly affect scattering-parameter measurements commonly used in RF laboratories. This ambiguity is highlighted to inform those working on revisions that is must be addressed. Finally, traditional uncertainty evaluation techniques for vector network analyser measurements in coaxial transmission lines are applied to rectangular metallic waveguide setups to investigate their success. Waveguide concerning frequencies up to 750 GHz are considered, covering E-band and higher which are being developed for future high-bandwidth communications. Although the uncertainty evaluation techniques work well for most waveguides tested, mechanical issues in WR-1.5 prohibits the feasibility of the technique.

\renewcommand{\baselinestretch}{1}\selectfont
\chapter*{Research Outcomes}
\vspace{-5mm}
\begin{refsection}
\section*{Publications}
\nocite{Votsi_2020, Salter_2018, Stant_2018_TMTT, Stant_2017, Stant_2016, Stant_2016_Coll}
\printbibliography[heading=none]
\end{refsection}
\vspace{-5mm}
\begin{refsection}
\section*{Presentations}
\nocite{Stant_2017_pgi, Stant_2017_feps, Stant_2016_pg, Stant_2016_Coll2}
\printbibliography[heading=none]
\end{refsection}
\chapter*{Acknowledgements}
I would like to thank my supervisory team: Peter Aaen, Nick Ridler and Marian Florescu. Peter and Nick gave me many opportunities throughout the project and were especially helpful and encouraging when problems arose. Marian took over as my primary supervisor while writing this dissertation and offered great advice and support to help me complete my work.

Martin Salter at the National Physical Laboratory (NPL), Teddington, who learned about nonlinear behavioural models and their extraction alongside me, always provided fruitful discussions and these helped me a great deal in my understanding of the science and mathematics. Martin also assisted with or performed some of the measurements in our publications.

I would also like to thank Daniel Stokes and other scientists in the electromagnetic measurements labs at NPL, who provided training and access to use their equipment, including the National primary standards for impedance in a variety of coaxial and waveguide media. Their welcoming and friendly nature also made every trip to NPL interesting, even when the schedule consisted entirely of repeated measurements!

I must extend huge thanks to Dylan Williams and his group at the National Institute for Standards and Technology (NIST) in Boulder, CO, USA. When I made contact with Dylan in regards to extending his existing software uncertainty framework he was immediately helpful and encouraging. He extended his upcoming UK trip to European Microwave Week and stayed at the university for an additional week to work with me expanding his software. The following year he invited me to visit NIST for three weeks to spend more time integrating behavioural model extraction into his software, and to characterise phase references used in my experiments. Both Dylan and his colleagues were excellent hosts and it was really exciting to work in an office with so much knowledge and experience. Special thanks also to Gustavo Avolio who performed the characterisation of our phase references, and also reviewed my final publication.

Away from my desk, my mother has provided me with a lot of support during my studies and put up with my various complaints. She has never stopped believing that I could finish my project and must feel quite vindicated now!

I am enormously grateful to my colleague and friend throughout my entire university career, Sean Gillespie. Together we were Peter's first PhD students at the University of Surrey, and experienced all the ups and downs of starting a new lab. We saw the university change a lot over the previous nine years, and I'm sure we will both look back in future with very fond memories of it.

Finally, last but not least, I would like to thank my fellow students and staff at the Advanced Technology Institute at the University of Surrey: Haris, Gemma, Jonas, Morgan, Steve, Matei, Tony, John, Grace, Julian, Jose, Kostis, Scott, James, Bob - the list goes on. Together they made my time as a postgraduate thoroughly enjoyable and I will miss them all!

\newpage
\tableofcontents
\addtocontents{toc}{~\hfill\textbf{Page}\par}

\newpage
\addcontentsline{toc}{chapter}{\listfigurename}
\listoffigures
\newpage
\addcontentsline{toc}{chapter}{\listtablename}
\listoftables

\glsaddall
\printunsrtglossary[title=List of Abbreviations]

%\clearpage
%\begingroup
%\thispagestyle{empty}
%\patchcmd{\chapter}{plain}{empty}{}{}
%\listoffigures
%\addtocontents{toc}{\contentsline{chapter}{\listfigurename}{}}
%\listoftables
%\addtocontents{toc}{\contentsline{chapter}{\listtablename}{}}
%\endgroup
%\clearpage

\end{document}
