\documentclass[../thesis/thesis.tex]{subfiles}
\renewcommand{\baselinestretch}{1.5}\selectfont
\graphicspath{{../figs/ch6-conc/}}
\begin{document}
\begin{refsection}
\chapter{Conclusion}

This dissertation has investigated several aspects of metrology for 5G applications. Due to the demanding performance requirements of the hardware used to implement this vision, it is crucial to underpin its development with accurate, confident measurements.

After introducing the work in Chapter 1, some fundamental knowledge on RF and microwave measurements was introduced in Chapter 2. This covered principles required to understand the contributions and methods of this research.

In Chapter 3, it was shown how core metrological principles are under continuous development. The statistical basis of these principles invites interpretative differences and so standardised guidance was developed \cite{GUM_2008}. However, supplements to this document have been published \cite{GUM_S1, GUM_S2} with alternative interpretations which cause an ambiguity in the guidance, of which efforts are underway to fix. Although for many types of measurement this ambiguity has little effect, for RF measurements, especially those involving S-parameters, the impact is significant. In new hardware implementations for future communications networks, massive numbers of antennas mean that measurements involving large numbers of s-parameters for a single device will become more common. For these measurements some of the guidance becomes unfeasible for typical manufacturer test processes, and even research applications. As part of this work, a review of this problem was published to aid the authors of the guidance documents in resolving this ambiguity \cite{Stant_2016}.

Chapter 4 introduced the application of measurement uncertainty to VNA and NVNA measurements. Of the two alternative approaches to quantifying calibration uncertainty, the residual method based on the widely-used ripple technique, was tested to see if it was applicable for use in waveguide. Future wireless communications are planned to use increasingly higher frequencies such as E-band (60--90 GHz), and frequencies around 300 GHz are being investigated for ultra-high bandwidth data transfer and streaming. To efficiently route these signals inside transceiver front-ends, waveguide is a common choice of transmission medium. The results of the investigation into the ripple technique in waveguide found that it was valid in waveguides up to 220 GHz, but mechanical alignment repeatability in waveguide at frequencies above this (WR-1.5) could cause the technique to fail \cite{Stant_2016_Coll,Stant_2017}. New developments in waveguide standards at this frequency should improve this repeatability and hopefully allow the ripple technique to provide uncertainty evaluations for measurements performed at these frequencies.

The main contribution of this work, however, was the first propagation of measurement uncertainty from nonlinear device measurements into a behavioural model, specifically X-parameters \cite{Stant_2018_TMTT}. This model is widely used in the amplifier design industry to allow engineers to simulate nonlinear device performance. The increasingly difficult design specifications required by 5G and other developing communications standards, combined with the commercial drive for first-pass design success, puts pressure on engineers to extract model parameters with accuracy and confidence. Until now, that confidence, quantified by measurement uncertainty, has not been rigorously evaluated.

The development of and results from this evaluation were presented in Chapter 5. The NIST Microwave Uncertainty Framework was introduced as an established and proven base from which to extend NVNA power wave uncertainties into a behavioural model. An invited secondment to NIST was used to improve this framework and extend it to work with the X-parameter nonlinear behavioural model. X-parameters were chosen as the model to implement for this work due to their popularity and compatibility with major vendors of both instrumentation and design simulation software, however it is possible to use the same framework to implement alternative behavioural models.

The uncertainty evaluation was demonstrated for a millimetre-wave amplifier. Over 300 sources of uncertainty were included, and all calibrations used traceable standards. The extracted model was subsequently used in an industry-standard circuit simulator as part of an amplifier design. The uncertainties were propagated through the simulations, using only built-in features, to allow impact of the measurement uncertainty in the extracted model to be seen in the final amplifier performance predictions. This new ability allows designers to review the quality of their model choices and develop confidence in their measurement instrumentation and processes.

\newpage
\section{Future Work}

Two of the three objectives of this dissertation have clear opportunities for future research.

The waveguide VNA uncertainty evaluation work presented in Chapter 4 highlighted issues with the repeatability of the mechanical alignment of the UG-387 flanges used on WR-1.5 waveguide. Since that investigation was completed, a new IEEE Standard has been published (and reviewed \cite{Ridler_2017}) which details new flange designs to increase the alignment repeatability. Therefore, it would be useful to repeat the evaluation of the uncertainty evaluation in WR-1.5 submillimetre-wave waveguide to investigate if the new flange designs make it valid.

The nonlinear behavioural model uncertainty evaluation would benefit greatly from the addition of another type of model. The Cardiff model \cite{Qi_2009} is a good candidate for this, as it is also commercially available and currently used in industry. With two models implemented in the framework, it would be possible to compare the sensitivities of the performance metrics of amplifiers simulated with them, which may provide new insights into their accuracy. This work could also review measurements of other nonlinear devices compatible with the behavioural models, such as mixers. On-wafer DUTs can also be included, using both calibration de-embedding techniques or the on-wafer absolute calibration standards under development \cite{Long_2016}.

Similar comparisons would be valuable between the different architectures of NVNA used to extract the models. The implementation developed in this work uses a mixer-based architecture, but measurement uncertainty propagated into behavioural models from sampler-based NVNAs should also be evaluated.

The reproducibility of behavioural model extraction, comparing the uncertainties in model parameters extracted from the same DUT at different labs using the same equipment is another opportunity for research using this new framework. Reproducibility studies are commonly used by both NMIs and industry to measure the variation in measurements across laboratories, which, combined with knowledge of the uncertainty tolerances of the application, can inform them where to focus on improving metrology \cite{ISO5725}. \looseness-1

Finally, an evaluation of behavioural model uncertainty using analytical propagation should be possible to develop, which can provide further verification against the numerical method presented here. This requires the derivation of Jacobian matrices relating the raw measured power waves to the model parameters. For mixer-based NVNAs, Jacobian matrices for calibrated power waves have already been published in \cite{Lin_2012}. Once those for the behavioural model have been derived, they can be cascaded to provide a complete analytical propagation of uncertainty. \looseness=-1

%\addcontentsline{toc}{section}{References}
%\printbibliography[title=References]
\end{refsection}
\end{document}
